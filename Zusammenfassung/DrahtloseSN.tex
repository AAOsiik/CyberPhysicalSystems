\documentclass[a4paper]{article}

%%% packages %%%%%%%%%%%%%%%%%%%%%%%%%%%%%%%%%%%%%%%%%%%%%%%%%%%%%%%%%%%%%%%%%
\usepackage{graphicx}
\usepackage{ngerman}
\usepackage{subcaption}
\usepackage{amsmath,amssymb}
\usepackage{alltt}
\usepackage{natbib} % please use \citep and \citet instead of \cite
\usepackage{tikz}
\usetikzlibrary{positioning,automata}
\usetikzlibrary{shapes.geometric}
\usetikzlibrary{shapes.arrows}
\usepackage{array}
\usepackage{hyperref}
\usepackage{xcolor}
\usepackage{listings}
\usepackage[export]{adjustbox}

\definecolor{dark-red}{rgb}{0.4,0.15,0.15}
\definecolor{dark-blue}{rgb}{0.15,0.15,0.8}
\definecolor{medium-blue}{rgb}{0,0,0.5}
\hypersetup{
	colorlinks, linkcolor={dark-red},
	citecolor={dark-blue}, urlcolor={medium-blue}
}

\graphicspath{{./figs/}}
\DeclareGraphicsExtensions{.pdf}

\setlength{\parindent}{0mm}

\usepackage{fancyhdr}

%%% %%%%%%%%%%%%%%%%%%%%%%%%%%%%%%%%%%%%%%%%%%%%%%%%%%%%%%%%%%%%%%%%%%%%%%%%%

\makeatletter
\newcommand{\seminar}{Drahtlose Sensornetze (SS 2019)}
\title{\textbf{Drahtlose Sensornetze: Zusammenfassung}}\let\Title\@title
\newcommand{\sTitle}{Drahtlose Sensornetze}
\newcommand{\AuthorName}{Alexander Osiik}
\author{\AuthorName\\
	\href{mailto:alexander.osiik@student.uni-luebeck.de}{alexander.osiik@student.uni-luebeck.de}\\
	\small \seminar\\
	%    \small Service Robotics Group\\
	\small Institute of Computer Engineering, University of L\"ubeck\\
}\let\Author\@author
\makeatother

\pagestyle{fancy}
\renewcommand{\footrulewidth}{0.4pt}
\lfoot{\seminar}
\cfoot{}
\rfoot{\thepage}
\lhead{\AuthorName}
\rhead{\sTitle}

%%% %%%%%%%%%%%%%%%%%%%%%%%%%%%%%%%%%%%%%%%%%%%%%%%%%%%%%%%%%%%%%%%%%%%%%%%%%

\begin{document}
	\maketitle

\section{Einführung}
\subsection{Sensorknoten?}
\begin{itemize}
	\item sind autonome Miniaturcomputer
	\item können: \begin{itemize}
		\item über Sensoren Wahrnehmen
		\item über Prozessoren verarbeiten
		\item über Funk kommunizieren
	\end{itemize}
\end{itemize}
\subsection{Sensornetz?}
\begin{itemize}
	\item drahtloses Netz aus vielen Sensorknoten
	\item weiträumig, langlebig
	\item erlaubt detaillierte Umweltbeobachtung
\end{itemize}
\subsection{Anwendungen?}
\begin{itemize}
	\item kann als wissenschaftliches Instrument benutzt werden\begin{itemize}
		\item Beobachtung von Tieren, Pflanzen, Umwelphänomene
	\end{itemize}
	\item Industrie\begin{itemize}
		\item Kontrolle von Infrastruktureinrichtungen
		\item Energiemanagement
	\end{itemize}
	\item Gesundheitswesen\begin{itemize}
		\item drahtlose Intensivstationene
		\item medizinische Forschung
	\end{itemize}
	\item Militär / Polizei\begin{itemize}
		\item Erkennung ``feindlicher'' Aktivitäten
	\end{itemize}
\end{itemize}
\subsection{Vorlesung}
Es soll Überblick zu Grundlegenden und einigen weiterführenden	Aspekten drahtloser Sensornetze verschaffen werden.\\
Es ist ein multidiziplinäres Gebiet:
\begin{itemize}
	\item Verteilte Systeme
	\item Informationssysteme
	\item Computer Systeme
	\item Eingebettete Systeme
	\item Ambient Computing
	\item Verteilte Algorithmen
\end{itemize}
\subsection{5G}
Handy verbindet sich mit Base Station Subsystem, dieses verbindet sich mit dem Network Switching System
\begin{itemize}
	\item Enhanced Mobile Breitband: Bisher wurden Frequenzen 1-6Gz genutzt. 5G nutzt nun bis 100GHz
	\item Ultra reliable und geringe Latenzen
	\item höhere Datenraten und mehr Frequenzen bei verringertem Energieverbrauch
\end{itemize}
\subsubsection{Kanalbündelung und Small Cells}
Kanalbündelung ist eine Bündelung der genutzten
Funkfrequenzbereiche eines Netzbetreibers
(Kanäle in einem Frequenzblock). Dies erlaubt es,
Datenrate pro Nutzer zu erhöhen.\\
Small Cell ist die Mobilfunkzelle mit geringer Sendeleistung, dementsprechend kleinem Versorgungsbereich. Der Radius liegt bei etwa 150m, die Sendeleistung ist gering
\subsubsection{MIMO: Massive Multiple Input Multiple Output}
Mehrantennensystem, das die zeitliche und räumliche Dimension nutzt.\\
Durch Space-Time-Coding wird die Zuverlässigkeit und Daterate gestigert (redundante Datenpakete)
\subsubsection{Versteigerung}
Man erhöht den Kaufpreis und spekuliert auf das Ausscheiden eines weiteren Anbieters. Falls der Anbieter spät ausscheidet, ist die investierte Summe im Endeffekt um ein vielfaches höher
\subsection{Betrachtete Netze}
die hier betrachteten Netze agieren anders, als z.B. 5G
\begin{itemize}
	\item Kommunikation geschieht nicht über Handy Netz
	\item Die Netzstruktur ist dezentral und selbst-organisiert
	\item hohe Energieeffizienz enorm wichtig für Sensorknoten -> Lange Lebenszeit
	\item Alles für die wissenschaftliche Messung / Überwachung
\end{itemize}

\section{Anwendungen}
Viele durch Sensornetze erreichte Optimierungen für das alltägliche Leben denkbar:
\begin{itemize}
	\item Neues Wissen ermöglichen
	\item Sicherheit erweitern
	\item Ressourcenmanagement
	\item Prävention von Fehlverhalten usw.
	\item[]
\end{itemize}

\par In folgende Kategorien kann unterteilt werden
\begin{itemize}
	\item Wissenschaftliches Instrument (``Macroscope'')
	\begin{itemize}
		\item Tiere, Pflanzen, Umweltphänomene
	\end{itemize}
	
	\item Industrielle Anwendungen
	\begin{itemize}
		\item Infrastruktureinrichtungen (Pipelines, Maschinen)
		\item Energiemanagement
	\end{itemize}
	
	\item Landwirtschaft
	\begin{itemize}
		\item Pflanzen (Wachstum, Reife, Bodenqualität)
		\item Tiere (Krankheiten, Fruchtbarkeit, virtuelle Zäune)	
	\end{itemize}
	
	\item Gesundheitswesen (``Body Sensor Networks'')
	\begin{itemize}
		\item drahtlose Intensivstation
		\item Verhaltensauffälligkeiten alter Menschen
		\item Lifestyle	
	\end{itemize}
	
	\item Militär / Polizei
	\begin{itemize}
		\item Erkennung, Klassifizierung, Lokalisierung des ``Bösen''	
	\end{itemize}
\end{itemize}

\subsection{Mikroklima}
Betrachtet wurden Redwoodbäume an der Küste Kaliforniens. Zur Motivation zählte die starke Variation und Dynamik klimatischer Bedingungen im Baum; kleine Wetterfronten bewegen sich entlang des Stamms. 
\subsubsection{System Überblick}
\textbf{Aufbau}
\begin{itemize}
	\item Sampling alle 5min
	\item 40-50 Knoten pro Baum
	\item Multi Hop Netz
	\item Messung von Temperatur, Feuchte, Sonneneinstrahlung
\end{itemize}
\textbf{Knoten:} Mote\\
\textbf{Netz:} TinyDB Sensor Netzwerk, Verbund über TASK Gateway, Funk über GPRS Modem\\

\subsubsection{Fazit}
\textbf{Beobachtung:} Trotz geringer Datenrate gab es einen hohen Datenverlust: 60\%  oder mehr\\

Ergebnis waren Zeitreihen pro Knoten, die gesamte Lebensdauer betrug 1,5 Monate. Pro Baum wurden 50 Knoten verwendet. Das Netz war \textbf{multihop}, \textbf{homogen} und \textbf{statisch}

\subsection{Brutverhalten}
Man möchte ein Modell für Brutpräferenzen des Wellenläufers erstellen. Dabei betrachtet man
\begin{itemize}
	\item Nestbelegung
	\item Klimatische Bedingungen in Höhlen
	\item Umweltbedingungen
\end{itemize}	
Die Beobachtung \textbf{muss} dezent erfolgen, da eine Abschreckung der Vögel blöd wäre.
	
\end{document}